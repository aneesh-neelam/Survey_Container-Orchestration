%
%  $Description: Survey on Orechestration$
%

\documentclass[10pt,twocolumn]{article}
\usepackage{times,mathptmx,fullpage}
% Allows you to put literal text into your document without having
% LaTeX try to interpret it.
\usepackage{verbatim}
% Sorts citations numerically and "combines" adjacent ones
\usepackage{cite}
% Allows you to individually label subfigures in a multi-part figure
\usepackage{subfigure}
% Allows the definition of macros that are smart about adding space
% after text in a macro
\usepackage{xspace}

% Redefine the percentage of the page that can be used for floats (figures,
% tables, etc.)
\renewcommand\floatpagefraction{.9}
\renewcommand\dblfloatpagefraction{.9}
\renewcommand\topfraction{.9}
\renewcommand\dbltopfraction{.9}
\renewcommand\bottomfraction{.9}
\renewcommand\textfraction{.1}
\setcounter{totalnumber}{10}
\setcounter{topnumber}{10}
\setcounter{dbltopnumber}{10}
\setcounter{bottomnumber}{10}

% Set values for float separation from text
\setlength{\floatsep}{1.5ex plus1.0ex minus 0.2ex}
\setlength{\dblfloatsep}{1.5ex plus1.0ex minus 0.2ex}
\setlength{\textfloatsep}{1.5ex plus1.0ex minus 0.2ex}
\setlength{\dbltextfloatsep}{1.5ex plus1.0ex minus 0.2ex}
\setlength{\abovecaptionskip}{0.5ex}
\setlength{\belowcaptionskip}{0.5ex}

% Don't allow widows or clubs - single lines at the start/end of a column
\widowpenalty=10000
\clubpenalty=10000

\newcommand{\latex}{\LaTeX\xspace}
\pagestyle{plain}

%-------------------------------------------------------------------------
\begin{document}

\title{A Survey on Virtual Machine and Container Orchestration}

\author{
Aneesh Neelam \\
\textit{Univ. of California, Santa Cruz} \\
\textit{aneelam@ucsc.edu}
}

\maketitle
\thispagestyle{empty}

\begin{abstract}

  Data center administrators and site-reliability engineers create virtual machines or containers to run their applications, with desired redundancy requirements and automated coordination between the replicas.
  This is called the orchestration of the various computing, storage and network resources of the data center.
  To automate the orchestration process, different tools have been developed in the past couple of decades.
  Each tool was designed differently, with different intents and priorities but they also have a lot in common.
  This survey is on the various orchestration tools available to data center administrators, site reliability engineers and software engineers;
  and aims to be a comprehensive guide that helps data center administrators in choosing the cloud platform and the orchestration tool.

\end{abstract}

%-------------------------------------------------------------------------
\section{Introduction}

With the advent and proliferation of virtualization, computing as a service has taken off.
Now there are multiple cloud service providers such as Google Cloud, Amazon Web Service, Microsoft Azure etc.
The operators of these cloud services typically have their own data centers, each consisting of thousands of physical machines connected via a high speed and very low latency network such as Infiniband (IB) ~\cite{intro_infiniband}.
Data center operators use various tools to automate the process of management, provisioning and scaling the resources for their customers.

Orchestration tools have been developed to ease the automation of operating a data center.
Tools such as VMware vCenter Orchestrator and Microsoft System Center Orchestrator, and OpenStack's Heat have been used to manage VMware vSphere, Microsoft Hyper-V or Linux kernel virtual machines.
Container Orchestration tools such as Docker Swarm, Kubernetes, Mesos have been developed to manage containers on mostly Linux machines, but also increasingly on Windows machines as well.

%-------------------------------------------------------------------------
\section{Background}

Please read the following instructions carefully.  You may use this
\latex document as a template if you're using \latex to prepare your
document.  There is also a Microsoft Word template if you prefer.
However, you may use any word processor you like, as long as you
follow the guidelines listed here.  Note that all manuscripts must be
in English.

%-------------------------------------------------------------------------
\subsection{Acceptable formats}

Your paper \emph{must} be submitted electronically.  The only acceptable
format for your final paper is PDF.

%-------------------------------------------------------------------------
\subsection{Margins and page numbering}

All printed material, including text, illustrations, and charts, must
be kept within a print area 6-7/8~inches (17.5\,cm) wide by 8-7/8
inches (22.54\,cm) high.  Nothing should be printed outside this area,
except for page numbers.

%------------------------------------------------------------------------
\subsection{Formatting your paper}

All text must be in a two-column format. The total allowable width of
the text area is 6-7/8~inches (17.46\,cm) wide by 8-7/8~inches
(22.54\,cm) high. Columns are to be 3-1/4~inches (8.25\,cm) wide, with
a 3/8~inch (0.95\,cm) space between them. The main title (on the first
page) should begin 1-3/8~inch (3.49\,cm) from the top edge of the
page. The second and following pages should begin 1.0~inch (2.54\,cm)
from the top edge. On all pages, the bottom margin should be 1-1/8
inches (2.86\,cm) from the bottom edge of the page for $8.5 \times
11$-inch paper; for A4 paper, approximately 1-5/8~inches (4.13\,cm)
from the bottom edge of the page.

%-------------------------------------------------------------------------
\subsection{Type-style and fonts}

Wherever Times is specified, Times Roman may also be used. If neither
is available on your word processor, please use the font closest in
appearance to Times that you have access to.  If possible, use only
Times, Helvetica, Courier, Symbol, and Dingbats fonts in your paper.
Any other fonts must be embedded in your document to ensure that it
will print properly.  \latex users: you \emph{must} use the
\texttt{times} package.  You can use the Computer Modern Roman font
if you prefer, as long as you're careful to ensure that you're using
the outline fonts so they show up well in Acrobat (PDF).

\subsection{Main title}

Center the title 1-3/8~inches (3.49\,cm) from the top edge of the
first page. The title should be in Times 14-point, boldface type.
Capitalize the first letter of nouns, pronouns, verbs, adjectives, and
adverbs; do not capitalize articles, coordinate conjunctions, or
prepositions (unless the title begins with such a word). Leave two
blank lines after the title.

\subsection{Author names and affiliations}

Author names are to be centered beneath the title and printed in Times
12-point, non-boldface type.  Affiliations and email addresses are to
be below each author's name, and set in Times 12-point italic type.
If all authors have the same affiliation, the affiliation can appear
centered below all authors' names.
This information is to be followed by two blank lines.

\subsection{Paper body}

The body of the paper (including the abstract) must be in a two-column
format.

The main text should be set in 10-point Times, single-spaced.  All
paragraphs should be indented 1~pica (approx. 1/6~inch or
0.422~cm). Make sure your text is fully justified---that is, flush
left and flush right. Please do not place any additional blank lines
between paragraphs.

\subsection{Figures and tables}

Figure and table captions can be in any font, as shown in
Figures~\ref{fig:shortcaption} and~\ref{fig:longcaption}.  Short
captions (single line) should be centered, as in
Figure~\ref{fig:shortcaption}.  Long captions should be aligned on
both sides and indented 1~pica on both left and right, as demonstrated
in Figure~\ref{fig:longcaption}.  Figure captions should be below
their figures, while table captions should be \emph{above} the table,
as shown in Table~\ref{tab:sample}.

\begin{table}
\caption{Sample table.}
\label{tab:sample}
\begin{center}
\begin{tabular}{|l|c|c|c|c|} \hline
\textbf{Level} & \textbf{Size} & \textbf{Style} & \textbf{Before} & \textbf{After} \\ \hline \hline
1 & 12 point & Bold & 1 line & 1 line \\ \hline
\end{tabular}
\end{center}
\end{table}

\begin{figure}
  \centerline{\fbox{Stuff that goes into the figure itself.}}
  \caption{Example of short caption.}
  \label{fig:shortcaption}
\end{figure}

\begin{figure}
  \centerline{\fbox{Things that go bump in the night.}}
  \caption{Example of long caption requiring more than one line.}
  \label{fig:longcaption}
\end{figure}


Figures and tables must be centered, and should be placed in a single
column if possible; however, large figures and tables may span the
entire page if necessary.  Page-spanning figures and tables should be
at the top or bottom of the page.  Figures and tables \emph{must} be
included as part of the PDF document---no cut and paste!

We will be printing the proceedings in black and white.  However,
color in your document is acceptable as long as the black and white
version is still readable and comprehensible; the CD-ROM will allow
readers to see color figures.

You can use subfigures, such as Figures~\ref{fig:subfigure1}
and~\ref{fig:subfigure2}, if you like, as shown in
Figure~\ref{fig:subfigures}.

\begin{figure}
  \subfigure[The first subfigure.]{
    \centerline{\fbox{Stuff that goes in the first subfigure.}}
    \label{fig:subfigure1}
  }
  \subfigure[The second subfigure.]{
    \centerline{\fbox{Stuff that goes in the second subfigure.}}
    \label{fig:subfigure2}
  }
  \caption{Sample use of subfigures.}
  \label{fig:subfigures}
\end{figure}

\section{First order headings}

First-order headings should be Times 12-point boldface, flush left,
with one blank line before, and one blank line after.  Only the first
word of first-, second-, and third-order headings should be
capitalized.  All numbers in the section number should be followed by
periods.

\subsection{Second order headings}

Second order headings should be Times 11-point boldface, flush left,
with one blank line before, and one after.

\subsubsection{Third order headings.}

If you require a third-order heading (we discourage it), use 10-point
Times, boldface, flush left, preceded by one blank line and followed by
one blank line.

%-------------------------------------------------------------------------
\subsection{Footnotes}

Please use footnotes sparingly\footnote{Or, better still, try to avoid
  footnotes altogether.  If you do use footnotes, place them at the
  bottom of the column on the page on which they are referenced using
  8-point single-spaced Times.}; instead, include necessary peripheral
observations in the text (within parentheses, if you prefer).

%-------------------------------------------------------------------------
\subsection{References}

List and number all bibliographical references in 9-point Times,
single-spaced, at the end of your paper. When referenced in the text,
enclose the citation number in square brackets, for
example~\cite{ex1}.  Multiple citations~\cite{ex1,ex2} should be in a
single set of square brackets.

%-------------------------------------------------------------------------
\subsection{Conclusions}

Please direct any questions to Professor Miller.  Keep in mind that
these instructions are taken from a conference, so soem of the
instructions, particularly those for procedures rather than
formatting, may not apply to you.

%-------------------------------------------------------------------------
\bibliographystyle{latex8}
\bibliography{orchestration}

\end{document}
